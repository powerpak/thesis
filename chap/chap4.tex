%% A chapter for my PhD dissertation
%% First author: Theodore Pak
%%
%% Must be included from main.tex.

\chapter{The PathogenDB software suite for genomic clinical microbiology \& epidemiology}
\label{chap:pathogendb}

\begin{quote}
\emph{Next-generation sequencing (NGS) technologies have reduced the cost of acquiring genomic data from active infections in hospitals, with the potential to rapidly characterize patient-to-patient transmission with extreme precision. However, there is no integrated software solution for converting NGS data into species identifications, phylogenies, and drug susceptibilities, with particularly few options for handling \emph{de novo} assemblies. A clinical application would ideally provide a unified pipeline that runs semi-automated analyses to inform infection control interventions. We developed a modular open-source software suite called PathogenDB that implements major functionalities needed for genomic clinical microbiology and pathogen surveillance. A central laboratory information management system runs on a standard open-source Linux/Apache/MySQL/PHP stack. A modular genomics workflow, PathogenDB-pipeline, automates de novo assembly, circularization, gene annotation, QC, epigenetic motif prediction. A comparative genomics module, PathogenDB-comparison, performs semi-automated phylogenetic analysis. Finally, a visualization suite, PathogenDB-viz, integrates phylogenies and epidemiological data into a ``live view'' of transmissions mapped to hospital locations. Thus far, PathogenDB-pipeline has been used to assemble and annotate 589 genomes from 7 species, and runs in <12 hours end-to-end. At an urban tertiary-care hospital, PathogenDB-comparison has genomically characterized one MRSA outbreak, two transmissions via solid organ transplant, and pseudo-outbreaks of \emph{S. maltophilia} and \emph{B. cepacia}. All three software packages are freely available on GitHub.
}
\end{quote}

- Need for this pipeline in pathogen surveillance (crib from chapter 1)
- Compare with GATK, Linderman pipeline for clinical genetics
- briefly recall tools available to do this
- What are the challenges? Reliability, speed/efficiency, reproducibility (engineering considerations). You want a 1d TAT max for bioinformatics so ID/Infection Control can act on the information.

\section{Implementation}

- Overview: fig from poster. Point out that this is our direct implementation of circle diagram from chap 1.
- Surveillance isolate collection. Recording isolate metadata.
    - introduce PathogenDB and relational DB structure (figure)
- pathogendb-pipeline: Assembly/annotation
    - Review steps taken and tools used for each, intermediate formats, etc.
    - Upgrade from homebrew circularization/RAST to more mature tools circlator/prokka
    - contig naming scheme
    - supports manual curation of assemblies
    - targets an IGB directory for diagnostic visualizations, incl. tracks for ChromoZoom.
- pathogendb-comparison: Comparative genomics
    - Review steps taken and tools used for each, intermediate formats, etc.
    - Note that mugsy is being replaced with Harvest/ParSNP
    - However, whole genome alignment for creating phylogeny 1000+ assemblies is inefficient for basic question of "is this isolate similar enough (within the threshold of likely transmission) to any of the past sequenced isolates"?
    - Therefore we have a streamlined pipeline for creating nucmer SNV distances that is O(n\^2)
        - Could further optimize by only doing comparisons with same MLST.
- Heatmap visualization
    - Only introduce implementation, PHP + d3 frontend based on JSON output of nodes + edges from comparison pipeline.

\section{Results and Discussion}

- Heatmap outputs
    - Heatmap view
    - Geospatial view
- Speed measurements
    - Assembly/annotation pipeline
    - Comparison pipeline too?

\section{Conclusion}

What could be concluded from this endeavor? You tell me.