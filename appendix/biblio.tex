%% Try to be more flexible about line breaks in links (esp. DOIs) to avoid poor justification

\setcounter{biburlnumpenalty}{100}
\setcounter{biburlucpenalty}{100}
\setcounter{biburllcpenalty}{100}

%% Prints the bibliography chapter. Note that this uses a biblatex macro, *not* natbib.
%% sloppypar allows more interword spacing to prevent overfull hboxes, since linebreaking citations
%% is much harder than normal text

\begin{sloppypar}
\printbibliography[heading=bibintoc]
\end{sloppypar}

%% After the bibliography, on the very last page, we add a colophon.

\clearpage

\vspace*{\fill}

\noindent\smallcaps{Colophon}

\vspace{1em}

\noindent{}This document was typset using \XeTeX, starting from a fork of Tiffany Tseng's \texttt{tufte-latex-mit} template,\footnote{\url{https://github.com/ttseng/tufte-latex-mit}} which is itself an amalgamation of components of MIT's \LaTeX{} thesis template\footnote{\url{http://web.mit.edu/thesis/tex/}} and the \texttt{tufte-latex} template.\footnote{\url{https://tufte-latex.github.io/tufte-latex/}} It unabashedly imitates the style of Aaron Turon's immaculately typeset PhD dissertation;\autocite{Turon2013} I hope he can forgive me for emulating an impressively optimal solution.

Body text is set in Minion Pro; monospaced text uses \texttt{Bitstream Vera Sans Mono}. {\fontfamily{lmr}\selectfont{}Latin Modern} is used for the pittance of math herein. {\chapterNumberNormalSize\textbf{Cochin}} is used for chapter numbers, while {\chapterTitle\itshape Palatino} stealthily supersedes Minion Pro for chapter titles.

Source code for the typesetting of this thesis is available online.\footnote{\url{https://github.com/powerpak/thesis}} I hereby disclaim any responsibility for the amount of procrastination said code enables in the course of writing your own dissertation.